\documentclass[11pt]{article}
\usepackage[T1]{fontenc}
\usepackage[utf8]{inputenc}
\usepackage{multirow}
\usepackage{setspace}
\usepackage{url}
%\usepackage{longtable}
\usepackage{natbib}
%\input{fancy}

\oddsidemargin  0.0in
\evensidemargin 0.0in
\textwidth      6.0in
%\headheight     0.5in
\headheight     0.0in
\topmargin      0.0in
\textheight 9.0in

\bibpunct{(}{)}{,}{a}{}{;}

\title{The Icelandic tagset}
\author{Hrafn Loftsson \\
        Department of Computer Science \\
        Reykjavik University \\
        e-mail: hrafn@ru.is \\
        }

\begin{document}
\hyphenation{Ice-Tagger Ice-Morphy Reykja-vik}
\label{firstpage}
\date{January 2007}
\maketitle
%\tableofcontents
%\newpage
%\begin{spacing}{1.5}

\section{Introduction}
\label{sec:tagset}
The main Icelandic tagset was created in the making of the Icelandic Frequency Dictionary (\emph{IFD}) corpus \citep{pin91}.
Due to the morphological richness of the Icelandic language, the main tagset is large and makes fine distinctions compared to related languages.
The rich inflections of an Icelandic word contribute more information about POS of surrounding words than is the case, for example, for English where word order is not as free.
The tagset consists of 700 possible tags, 639 tags of which appear in the \emph{IFD} corpus.
%: 192 noun tags, 163 pronoun tags, 144 adjective tags, 82 verb tags, 27 numeral tags, 24 article tags, 16 punctuation tags, 9 adverb/preposition tags, 3 conjunction tags and 1 tag for foreign words and words not analysed, respectively.

We can illustrate the preciseness of the tags by examining the semantics of a tag.
Each character in the tag has a particular function.
The first character denotes the word class.
For each word class there is a predefined number of additional characters (at most six) which describe morphological features, like \emph{gender}, \emph{number} and \emph{case} for nouns; \emph{degree} and \emph{declension} for adjectives; \emph{voice}, \emph{mood} and \emph{tense} for verbs, etc.

Tables \ref{tab:semanticsNounAdj} and \ref{tab:semanticsVerb} show the semantics of the noun and the adjective tags, and the semantics of the verb tags, respectively.
Consider, for example, the tag ``\emph{nken}''.  The first letter, ``\emph{n}'', denotes the word class ``\emph{nafnor{\dh}}'' (noun), the second letter, ``\emph{k}'', denotes the gender ``\emph{karlkyn}'' (masculine), the third letter, ``\emph{e}'', denotes the number ``\emph{eintala}'' (singular) and the last letter, ``\emph{n}'', denotes the case ``\emph{nefnifall}'' (nominative case).

\begin{table}
\begin{tabular}{lll}
\hline
\hline
Char & Category/ & Symbol -- semantics \\
\# & Feature \\
\hline
1 & Word class & {\bf n}--noun, {\bf l}--adjective  \\
2 & Gender & {\bf k}--masculine, {\bf v}--feminine, {\bf h}--neuter, {\bf x}--unspecified  \\
3 & Number & {\bf e}--singular, {\bf f}--plural \\
4 & Case & {\bf n}--nominative, {\bf o}--accusative, {\bf {\th}}--dative, {\bf e}--genitive \\
5 & Article & {\bf g}--with suffixed article \\
5 & Declension & {\bf v}--strong, {\bf s}--weak  \\
6 & Proper noun & {\bf m}--person, {\bf {\"o}}--place, {\bf s}--other \\
6 & Degree & {\bf f}--positive, {\bf m}--comparative, {\bf e}--superlative \\
\hline
\hline
\end{tabular}
\caption{The semantics of the tags for nouns and adjectives.}
\label{tab:semanticsNounAdj}
\end{table}

\begin{table}
\begin{tabular}{lll}
\hline
\hline
Char & Category/ & Symbol -- semantics \\
\# & Feature \\
\hline
%\pagebreak
1 & Word class & {\bf s}--verb (except for past participle) \\
2 & Mood & {\bf n}--infinitive, {\bf b}--imperative, {\bf f}--indicative, {\bf v}--subjunctive, \\
  & & {\bf s}--supine, {\bf l}--persent participle \\
3 & Voice & {\bf g}--active, {\bf m}--middle  \\
4 & Person & {\bf 1}--$1^{st}$ person, {\bf 2}--$2^{nd}$ person, {\bf 3}--$3^{rd}$ person,  \\
5 & Number & {\bf e}--singular, {\bf f}--plural \\
6 & Tense & {\bf n}--present, {\bf þ}--past\\
\hline
\hline
\end{tabular}
\caption{The semantics of the tags for verbs.}
\label{tab:semanticsVerb}
\end{table}

To give another example, consider the phrase ``\emph{fallegu hestarnir hoppu{\dh}u}'' (the beautiful horses jumped).
The corresponding tag for ``\emph{fallegu}'' is ``\emph{lkenvf}'', denoting adjective, masculine, singular, nominative, weak declension, positive;
the tag for ``\emph{hestarnir}'' is ``\emph{nkfng}'', denoting noun, masculine, plural, nominative with suffixed definite article, and the tag for ``\emph{hoppu{\dh}u}'' is ``\emph{sfg3f{\th}}'', denoting verb, indicative mood, active voice, 3-rd person, plural and past tense.
Note the agreement in gender, number and case between the adjective and the noun, and the agreement in person and number between the adjective/noun and the verb.

A complete description of the tagset can be found in the Appendix \ref{appendix:tagset}.

\begin{spacing}{1.0}
%\addcontentsline{toc}{section}{References}
\bibliographystyle{abbrvnat}
\bibliography{ref}
\end{spacing}

\newpage
\begin{spacing}{1.0}
\appendix
%\addcontentsline{toc}{section}{Appendix}
\section{The Icelandic tagset}
\label{appendix:tagset}
\begin{table}[h]
\begin{center}
{\scriptsize
\caption{The Icelandic tagset}
%\begin{longtable}{lll}
\begin{tabular}{lll}
\hline
\hline

Char\# & Category/Feature & Symbol -- semantics \\
\hline
%\endhead
1 & Word class & {\bf n}--noun \\
2 & Gender & {\bf k}--masculine, {\bf v}--feminine, {\bf h}--neuter, {\bf x}--unspecified  \\
3 & Number & {\bf e}--singular, {\bf f}--plural \\
4 & Case & {\bf n}--nominative, {\bf o}--accusative, {\bf {\th}}--dative, {\bf e}--genitive  \\
5 & Article & {\bf g}--with suffixed definite article \\
6 & Proper noun & {\bf m}--person name, {\bf {\"o}}--place name, {\bf s}--other proper name \\
\hline
1 & Word class & {\bf l}--adjective \\
2 & Gender & {\bf k}--masculine, {\bf v}--feminine, {\bf h}--neuter \\
3 & Number & {\bf e}--singular, {\bf f}--plural \\
4 & Case & {\bf n}--nominative, {\bf o}--accusative, {\bf {\th}}--dative, {\bf e}--genitive  \\
5 & Declension & {\bf s}--strong declension, {\bf v}--weak declension, {\bf o}--indeclineable  \\
6 & Degree & {\bf f}--positive, {\bf m}--comparative, {\bf e}--superlative \\
\hline
1 & Word class & {\bf f}--pronoun \\
2 & Subcategory & {\bf a}--demonstrative, {\bf b}--reflexive, {\bf e}--possessive, {\bf o}--indefinite, \\
  & & {\bf p}--personal, {\bf s}--interrogative, {\bf t}--relative  \\
3 & Gender/Person & {\bf k}--masculine, {\bf v}--feminine, {\bf h}--neuter/{\bf 1}--$1^{st}$ person, {\bf 2}--$2^{nd}$ person \\
4 & Number & {\bf e}--singular, {\bf f}--plural \\
5 & Case & {\bf n}--nominative, {\bf o}--accusative, {\bf {\th}}--dative, {\bf e}--genitive  \\
\hline
1 & Word class & {\bf g}--article \\
2 & Gender & {\bf k}--masculine, {\bf v}--feminine, {\bf h}--neuter \\
3 & Number & {\bf e}--singular, {\bf f}--plural \\
4 & Case & {\bf n}--nominative, {\bf o}--accusative, {\bf {\th}}--dative, {\bf e}--genitive  \\
\hline
1 & Word class & {\bf t}--numeral \\
2 & Category & {\bf f}--alpha, {\bf o}--numeric, {\bf p}--percentage \\
3 & Gender & {\bf k}--masculine, {\bf v}--feminine, {\bf h}--neuter  \\
4 & Number & {\bf e}--singular, {\bf f}--plural \\
5 & Case & {\bf n}--nominative, {\bf o}--accusative, {\bf {\th}}--dative, {\bf e}--genitive  \\
\hline
%\pagebreak
1 & Word class & {\bf s}--verb (except for past participle) \\
2 & Mood & {\bf n}--infinitive, {\bf b}--imperative, {\bf f}--indicative, {\bf v}--subjunctive, \\
  & & {\bf s}--supine, {\bf l}--persent participle \\
3 & Voice & {\bf g}--active, {\bf m}--middle  \\
4 & Person & {\bf 1}--$1^{st}$ person, {\bf 2}--$2^{nd}$ person, {\bf 3}--$3^{rd}$ person,  \\
5 & Number & {\bf e}--singular, {\bf f}--plural \\
6 & Tense & {\bf n}--present, {\bf þ}--past\\
\hline
1 & Word class & {\bf s}--verb (past participle) \\
2 & Mood & {\bf þ}--past participle\\
3 & Voice & {\bf g}--active, {\bf m}--middle  \\
4 & Gender & {\bf k}--masculine, {\bf v}--feminine, {\bf h}--neuter \\
5 & Number & {\bf e}--singular, {\bf f}--plural \\
6 & Case & {\bf n}--nominative, {\bf o}--accusative, {\bf {\th}}--dative, {\bf e}--genitive  \\
\hline
1 & Word class & {\bf a}--adverb and preposition \\
2 & Category & {\bf a}--does not govern case, {\bf u}--exclamation, \\
  & & {\bf o}--governs accusative, {\bf þ}--governs dative, {\bf e}--governs genitive \\
3 & Degree & {\bf m}--comparative, {\bf e}--superlative \\
\hline
1 & Word class & {\bf c}--conjunction \\
2 & Category & {\bf n}--sign of infinitive, {\bf t}--relative conjunction, \\
\hline
1 & Word class & {\bf e}--foreign word\\
\hline
1 & Word class & {\bf x}--unanalyzed word \\
\hline
\hline
%\end{longtable}
\end{tabular}
}
\end{center}
\end{table}
\end{spacing}

%\end{spacing}
\end{document}
